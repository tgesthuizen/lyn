\documentclass[12pt, a4paper]{article}

\usepackage{generic-article}

\author{Tim Gesthuizen}
\title{Short Programming Project: A compiler for embedded systems}

\begin{document}

\maketitle

\section{Motivation}

Modern compilers tend to be very large projects that require lots of
memory when being run.
Only the quality of the generated code seems to be of importance.

Other programming languages use a more dynamic model where developers
can try new code and the REPL and then slowly build the programs
source code from already proven to work snippets.
In addition this kind of development has much shorter feedback loops
and allows for more interactive and responsive debugging.
On the other hand the ability to change definitions on the go imposes
optimization barriers for the compiler and leads to performance
penalties.

This project experiments with writing a small compiler for an embedded
system, possibly running on said embedded system, and enabling a
hybrid of the two development forms.
As embedded platforms often have a long waiting time for compiling and
flashing the full software this would be benefitial for exploratory
programming and aid with debugging.

\section{Project definition and goals}

Because the project was deemed rather ambitious it is split into
stages which each will each result in a piece of software.

Initially, the language to be compiled needs to be defined.
The main goal of this project is to write a compiler in C++ that
compiles this minimal language for the ARM architecture using the
Thumb instruction set.
Code must be compile- and runable on the  target platform\footnote{The
  target platform is yet to be determined but can be any ARM-based
  platform.}.
It would be nice to get the compiler running on the target platform,
but this might be too much work for the course.
If there is even more time remaining it can be spent with analysis of
how much ressources are used by the compiler, how good the generated
code is, extending the compiled language or adding optimization
passes.

In order to keep track of the progress the following milestones for
the main goal are defined:

\begin{enumerate}
\item Defining the target language
\item The language is parsed by the compiler
\item Semantic analysis is done by the compiler
\item The code is translated into an intermediate representation by
  the compiler
\item Program code is generated from the intermediate representation
\item A first working program was run on the target platform
\item The compile-run process is streamlined
\end{enumerate}

For the optional goal the following milestones are defined:
\begin{enumerate}
\item Firmware code in C/C++ can be compiled for the target platform
\item The backend of the compiler is altered to emit code into memory
  in a runable form
\item The compiler was successfully compiled for the target platform.
\item The first program was compiled and run on the target platform
\item The compile-run process is streamlined
\end{enumerate}

\section{Further information}

Gunnar Kudrjavets is supervising the project.

Regular meetings will be set up to report on the current status and
adapt the scope of the project to the actual progress.

\end{document}