\documentclass[12pt, article]{article}

\author{Tim Gesthuizen}
\title{Logbook}

\setlength{\parindent}{0ex}
\setlength{\parskip}{1.5ex}

\begin{document}

\maketitle

\section{2022-11-17}

\subsection{What happened today?}

\begin{itemize}
\item Parsing works okay for simple programs
\item The scopify and typecheck passes work as well
\item The name of the language is decided to be ``lyn'' for now.
  The short name saves typing and having a name is better than
  bikeshedding for days what would be a good one.
\item Added this document to have a time-representing documentation of
  the project.
\end{itemize}

\subsection{Some questions came up}

\begin{itemize}
\item What would be a good intermediate format to translate to ARM
  instructions? (Suspected to be K-normal forms, like mincaml uses)
\item What would be a good separation for the project?
\item What would be a good first target platform?
\item How would a hand-written recursive descent parser compare to the
  GNU Bison one?
\end{itemize}

\subsection{New ideas}

I guess it would make most sense to develop the compiler in three
pulls:
\begin{enumerate}
\item Build a version that has no primitives nor register allocation
  whatsoever.
  This is easy to implement and a nice comparison for later
  developments.
  Code generated by this version will be trivial and horrible.
\item Then add primitive handling and get rid of some superflous calls
  that can be replaced with simple machine instructions.
\item At last add register allocation.
\end{enumerate}

Maybe step 2 and 3 can be swapped, although it seems safest to
approach it in the way written down.

\subsection{Notes of today}

The amount of memory allocations being done is worrying.
Especially the type system probably needs some reworking before it can
be ported to embedded systems.
Also types are never ever freed by the program yet, but don't tell
anyone.
Right now it seems that types can be dropped after they are checked so
I might just allocate them in a separate heap and drop it altogether
once IR translation has been done.

\end{document}